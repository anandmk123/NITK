\documentclass[conference]{IEEEtran}
\usepackage[utf8]{inputenc}
\usepackage{cite}
\usepackage{tabularx} % Required for the table
\usepackage{array}    % Required for custom column types
\usepackage{longtable}
\usepackage{booktabs}
\usepackage{adjustbox} % To allow scaling of tables
\usepackage{caption}
\usepackage{stfloats}
\usepackage{authblk}

\begin{document}

\title{Power, Area and Thermal Prediction in 3D Network-on-Chip using Machine Learning}

\author{Abhijith C, Anand M K \\

Department of Computer Science and Engineering \\ 
	National Institute of Technology Karnataka (NITK) \\ 
	Surathkal, India\\
Email: \{abhijithc.242cs003, anandmk.242cs008\}@nitk.edu.in}


\maketitle

\section{Proposed Methodology}

\hspace{1cm}

Machine learning (ML) algorithms are widely used in various real-time predictions, such as energy consumption prediction and weather forecasting. The Power, Area and Temperature (PAT) prediction of Network-on-Chip can also leverage the ability of machine learning models. However, the lack of availability of a proper public dataset is a crucial issue. In addition, most existing studies focus on power, area or thermal analysis independently. Frameworks involving simultaneous analysis of all three parameters are rare in current research. 

This work proposes a hybrid model that combines ML and DL algorithms. The ML component of the hybrid model uses algorithms such as linear regression and decision trees to predict area and power, which have a linear relationship with NoC parameters. The DL component of the model uses convolutional neural networks (CNNs) that learn complex nonlinear relationships in NoC to predict temperature. The algorithms can predict PAT values for unseen input NoC configuration based on their learning. The prediction of both models is combined to produce a single output in the hybrid model. 


\end{document}
